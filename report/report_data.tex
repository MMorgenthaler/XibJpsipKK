\section{Data acquisition}
\noindent
The data which was used in this study was obtained from two sources. First of all the simulator \textit{RapidSim} was used to generate the desired decay channel for different option parameters. Alternatively real LHCb data from Run I and Run II, namely 2011 to 2012 and 2015 to 2018 was cut and used in the study. In the following it's described how those datasets were produced and which one was used for the investigation of the  

\subsection{RapidSim}
\noindent
The data which was used at first was generated by \textit{RapidSim}. It's a quick and dirty way to simulate hadronic data. In the first attempts rather generic settings were used. They mirrored the settings used by Julian Bollig to generate data. The chosen settings were: 

\begin{enumerate}
	\item[geometry: LHCb]
	\item[pid: LHCbGenericPID]
	\item[smear: LHCbGeneric]
\end{enumerate}

\begin{enumerate}
	\item[geometry: LHCb]
	\item[pid: LHCbGenericPID]
	\item[acceptance: AllIn]
	\item[etaRange: 1 8]
	\item[energy: 7]
	\item[smear: LHCbGeneric]
\end{enumerate}

\subsection{CERN data}